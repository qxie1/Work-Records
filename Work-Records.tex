\documentclass[a4paper,10pt]{article}
\usepackage[utf8x]{inputenc}
\usepackage{color}

%opening
\title{implementation of Magnetic particles in LBM}
\author{Qingguang Xie}

\begin{document}

\maketitle

\begin{abstract}

\end{abstract}
\newpage

\section{16.12.2013-20.12.2013}
The conflict between the applied torque and response of the particles.
\begin{itemize}
 \item The difference between direction of applied magnetic filed and the direction of movements of magnetic partilces.
       The force direction is right, however, the torque direction is opposite.
       In the code it is right hand orientation, but after applying a positive torque, the rotation is negative.
  \item Now the quarternion aggrees well with orientation. 
        The hidden error may be in rotation matrix, but I checked the code line by line, there is no error on the signs.
   \item  More information about the bug, under constant torque the output angular velocity direction is dependent on the orientation.
          \textcolor{red}{The reason is found. The body-fixed angular veloctiy is dumped. 
          I convered the body-fixed angular velocity to space-fixed angular velocity, then output 
          it, now for constant torque it works well.}\\
          However, now it behaves in left hand rule.
   \item  With constant force, the motion of particle is correct. With linear external field, the motion of magnetic particle is correct. 
    \item The interaction between two particles are correct, except that the torque system is left hand rule.
\textcolor{red}{}
\end{itemize}
Balance between capillary force and magnetic force.~\cite{Lumay2013}
\begin{itemize}
 \item Apply constant force on particles, constant external field vertical to interface making magnetic particle repulsive each other.
      further more, add a magnetic filed parallel to interface
 \item Parameter $fix\_vx$ is used to fix x coordinate of particles for $fix\_vx\_t$ time steps, then the particle is released.
     The purpose is that let the interface is fully deformed under constant force
     before entering play with magnetic force.
 \item A group of parameters for two particles work well. 
      Domain size $128*64*64$, $R=5$,distance between two particles $dr=30$, constant force $F=0.5$, magtitude of magnetic dipole $m_i=3e5$, constang magnetic filed $B=[0,1e-4,0]$
      The Scallop theorem states that to achieve propulsion at low Reynolds number in Newtonian fluids a swimmer must deform in a way that is not invariant under time-reversal. 
      Nonreciprocal motion is only possible for $N>2$, a pair of particles is a poor swimmer~\cite{Lumay2013}. 
      However, two or more dumb-bells, with mutual phase differences can swimm~\cite{Alexander2008}.  
 \item  Three particles, 
       Domain size $128*64*128$, $R=5$,distance between two particles $dr=30$,\\ 
       magtitude of magnetic dipole $m_i=3e5$, constang magnetic filed $B=[0,1e-4,0]$.\\
       1.With constant force $F=0.5$,the particles detach the interface.  \\
       2.Decrease force to $F=0.2$, and $m_i=2e5$,$dr=20$,$h_c=31$ failed, because particle move out of cut off\\
       3.Change domain size $128*128*128$, $F=0.5$, $m_i=3e5$, $dr=30$,$h_c=61$,
      The spring-linked three-particle swimmer is studied by LBM~\cite{Pickl2012}.
      Symmetry breaking in a few-body system with magnetocapillary interactions is investigated~\cite{Lumay2012}
\end{itemize}
Interface deformation and magnetic torque
\begin{itemize}
 \item A magnetic field which is vertical to interface is applied, the particle rotates following the magnetic field, 
       when the particle orientation is parallel to the applied external field, the magnetic torque vanishes. then due to the deformation 
of interface, the particle rotates back releasing the free energy of the interface, thus causing an oscillation.
\textcolor{red}{This seems interesting, may be used in the future.}
\end{itemize}
Conversion from LBM unit to SI unit
\begin{itemize}
 \item
      $C_s = 1482.35 m/s $\\
      $\nu_r = 1.0 \times 10^{-6} m^2/s$\\
      $D=2.8 \times 10^{-6} m$\\
      $c_s = \frac{1}{\sqrt{3}}$, $D=10$ \\
      $\delta x= L/N = 2.8\times 10^{-7} m$  \\      
      $\delta t = c_s/C_s * dx = 1.09 \times 10^{-10} s$ \\
      $\nu = (dx*dx)/dt \times {c_s}^2\times (\tau -0.5) $\\
      $\rho =2, m= \frac{4}{3}\pi\rho R^3$       
         
\end{itemize}
Rotation of single particle at the interface
\begin{itemize}
 \item  Domain size $128*64*128$, $R=10$, $t_z=0.5$
\end{itemize}
Further Plan 
\begin{itemize}
 \item Implementation of superamagnetic particles
\end{itemize}

\section{Bibliography}
\bibliographystyle{unsrt}
\bibliography{work-record}

\end{document}
